%%
%% This is file `sample-sigconf.tex',
%% generated with the docstrip utility.
%%
%% The original source files were:
%%
%% samples.dtx  (with options: `sigconf')
%% 
%% IMPORTANT NOTICE:
%% 
%% For the copyright see the source file.
%% 
%% Any modified versions of this file must be renamed
%% with new filenames distinct from sample-sigconf.tex.
%% 
%% For distribution of the original source see the terms
%% for copying and modification in the file samples.dtx.
%% 
%% This generated file may be distributed as long as the
%% original source files, as listed above, are part of the
%% same distribution. (The sources need not necessarily be
%% in the same archive or directory.)
%%
%% The first command in your LaTeX source must be the \documentclass command.
\documentclass[sigconf]{acmart}

%%
%% \BibTeX command to typeset BibTeX logo in the docs
\AtBeginDocument{%
  \providecommand\BibTeX{{%
    \normalfont B\kern-0.5em{\scshape i\kern-0.25em b}\kern-0.8em\TeX}}}

%% Rights management information.  This information is sent to you
%% when you complete the rights form.  These commands have SAMPLE
%% values in them; it is your responsibility as an author to replace
%% the commands and values with those provided to you when you
%% complete the rights form.
\setcopyright{acmcopyright}
\copyrightyear{2018}
\acmYear{2018}
\acmDOI{10.1145/1122445.1122456}

%% These commands are for a PROCEEDINGS abstract or paper.
\acmConference[Woodstock '18]{Woodstock '18: ACM Symposium on Neural
  Gaze Detection}{June 03--05, 2018}{Woodstock, NY}
\acmBooktitle{Woodstock '18: ACM Symposium on Neural Gaze Detection,
  June 03--05, 2018, Woodstock, NY}
\acmPrice{15.00}
\acmISBN{978-1-4503-XXXX-X/18/06}


%%
%% Submission ID.
%% Use this when submitting an article to a sponsored event. You'll
%% receive a unique submission ID from the organizers
%% of the event, and this ID should be used as the parameter to this command.
%%\acmSubmissionID{123-A56-BU3}

%%
%% The majority of ACM publications use numbered citations and
%% references.  The command \citestyle{authoryear} switches to the
%% "author year" style.
%%
%% If you are preparing content for an event
%% sponsored by ACM SIGGRAPH, you must use the "author year" style of
%% citations and references.
%% Uncommenting
%% the next command will enable that style.
%%\citestyle{acmauthoryear}

%%
%% end of the preamble, start of the body of the document source.
\begin{document}

%%
%% The "title" command has an optional parameter,
%% allowing the author to define a "short title" to be used in page headers.
\title{The Name of the Title is Hope}

%%
%% The "author" command and its associated commands are used to define
%% the authors and their affiliations.
%% Of note is the shared affiliation of the first two authors, and the
%% "authornote" and "authornotemark" commands
%% used to denote shared contribution to the research.
\author{Ben Trovato}
\authornote{Both authors contributed equally to this research.}
\email{trovato@corporation.com}
\orcid{1234-5678-9012}
\author{G.K.M. Tobin}
\authornotemark[1]
\email{webmaster@marysville-ohio.com}
\affiliation{%
  \institution{Institute for Clarity in Documentation}
  \streetaddress{P.O. Box 1212}
  \city{Dublin}
  \state{Ohio}
  \postcode{43017-6221}
}

\author{Lars Th{\o}rv{\"a}ld}
\affiliation{%
  \institution{The Th{\o}rv{\"a}ld Group}
  \streetaddress{1 Th{\o}rv{\"a}ld Circle}
  \city{Hekla}
  \country{Iceland}}
\email{larst@affiliation.org}

\author{Valerie B\'eranger}
\affiliation{%
  \institution{Inria Paris-Rocquencourt}
  \city{Rocquencourt}
  \country{France}
}

\author{Aparna Patel}
\affiliation{%
 \institution{Rajiv Gandhi University}
 \streetaddress{Rono-Hills}
 \city{Doimukh}
 \state{Arunachal Pradesh}
 \country{India}}

\author{Huifen Chan}
\affiliation{%
  \institution{Tsinghua University}
  \streetaddress{30 Shuangqing Rd}
  \city{Haidian Qu}
  \state{Beijing Shi}
  \country{China}}

\author{Charles Palmer}
\affiliation{%
  \institution{Palmer Research Laboratories}
  \streetaddress{8600 Datapoint Drive}
  \city{San Antonio}
  \state{Texas}
  \postcode{78229}}
\email{cpalmer@prl.com}

\author{John Smith}
\affiliation{\institution{The Th{\o}rv{\"a}ld Group}}
\email{jsmith@affiliation.org}

\author{Julius P. Kumquat}
\affiliation{\institution{The Kumquat Consortium}}
\email{jpkumquat@consortium.net}

%%
%% By default, the full list of authors will be used in the page
%% headers. Often, this list is too long, and will overlap
%% other information printed in the page headers. This command allows
%% the author to define a more concise list
%% of authors' names for this purpose.
\renewcommand{\shortauthors}{Trovato and Tobin, et al.}

%%
%% The abstract is a short summary of the work to be presented in the
%% article.
\begin{abstract}
\section{Abstract}
When discussing the privacy of blockchains or Distributed Ledger Technologies, the
focus of the analysis is often on the consensus layer of the blockchain.
Another major factor for a comprehensive privacy analysis is
the network layer, which defines how messages are passed between peers.
On this layer many privacy sensitive informations like public IP addresses of users and connections between the peers
of the network are used and can potentially be recorded by malicious parties.
In this report we will do a simulation based, quantitative privacy analysis of the Kademlia inspired
network overlay Kadcast, that can be used for efficient broadcasting e.g. in Bitcoin or other blockchain based technologies.
The focus of the anaylsis is the usage of Kadcast for broadcasting transactions and the evaluation of an adversary's capability
of linking observed transactions to IP addresses.
  %A clear and well-documented \LaTeX\ document is presented as an
  %article formatted for publication by ACM in a conference proceedings
  %or journal publication. Based on the ``acmart'' document class, this
  %article presents and explains many of the common variations, as well
  %as many of the formatting elements an author may use in the
  %preparation of the documentation of their work.
\end{abstract}

%%
%% The code below is generated by the tool at http://dl.acm.org/ccs.cfm.
%% Please copy and paste the code instead of the example below.
%%
\begin{CCSXML}
<ccs2012>
 <concept>
  <concept_id>10010520.10010553.10010562</concept_id>
  <concept_desc>Computer systems organization~Embedded systems</concept_desc>
  <concept_significance>500</concept_significance>
 </concept>
 <concept>
  <concept_id>10010520.10010575.10010755</concept_id>
  <concept_desc>Computer systems organization~Redundancy</concept_desc>
  <concept_significance>300</concept_significance>
 </concept>
 <concept>
  <concept_id>10010520.10010553.10010554</concept_id>
  <concept_desc>Computer systems organization~Robotics</concept_desc>
  <concept_significance>100</concept_significance>
 </concept>
 <concept>
  <concept_id>10003033.10003083.10003095</concept_id>
  <concept_desc>Networks~Network reliability</concept_desc>
  <concept_significance>100</concept_significance>
 </concept>
</ccs2012>
\end{CCSXML}

\ccsdesc[500]{Computer systems organization~Embedded systems}
\ccsdesc[300]{Computer systems organization~Redundancy}
\ccsdesc{Computer systems organization~Robotics}
\ccsdesc[100]{Networks~Network reliability}

%%
%% Keywords. The author(s) should pick words that accurately describe
%% the work being presented. Separate the keywords with commas.
\keywords{datasets, neural networks, gaze detection, text tagging}

%% A "teaser" image appears between the author and affiliation
%% information and the body of the document, and typically spans the
%% page.
\begin{teaserfigure}
  %\includegraphics[width=\textwidth]{sampleteaser}
  \caption{Seattle Mariners at Spring Training, 2010.}
  \Description{Enjoying the baseball game from the third-base
  seats. Ichiro Suzuki preparing to bat.}
  \label{fig:teaser}
\end{teaserfigure}

%%
%% This command processes the author and affiliation and title
%% information and builds the first part of the formatted document.
\maketitle

\section{Introduction}
So called ``Cryptocurrencies'' and other blockchain based technologies are gaining
increasing attention since the emergence of Bitcoin, both in the academic community and in media []. \\
%.[hier bessere ueberleitung]
These cryptocurrencies, like Bitcoin can be used analogous to fiat money to transfer ``coins''
from a sender to a receiver. In Bitcoin, new coins can be ``mined'' by investing computing power.
Unlike a fiat money transaction, which is generally handled by a bank, there is no central instance
that authorizes Bitcoin transactions. Instead Bitcoin is a peer-to-peer based system that is run publicly on the internet
and allows all participants to achieve consensus on a single valid transaction history without the need for a trusted third party.
Users can participate in the network via pseudonymous identities and there is no inherent link between a pseudonymous public identifier and a natural person,
albeit deanonymization attacks can be performed to potentially link the public key of a user to an IP address or even a real name [].
To store the transaction history, Bitcoin uses an append-only, distributed ledger (the so called ``blockchain'')
and consensus on the state of the ledger is periodically reached
by the participants of the network via a distributed consensus algorithm. [for detailed explanation refer to nakamoto paper]
Since payments and money flow comprise of very privacy sensitive information, and everyone can join the Bitcoin peer-to-peer network
and access the transaction history, the privacy of Blockchains has been an active research area since ...[].
There is a large amount of scientific literature [specifically] on the privacy of the consensus layer of
blockchains [], describing weaknesses [e.g using heuristics to construct entity graph, ...] and algorithms to mitigate privacy issues [mixing, coinjoin, zk-snarks, ...] [bib]. \\
Yet another important factor for a comprehensive privacy analysis is the network layer, which
defines how messages are passed between participants of the network.
Because of the open nature of public blockchains, any adversery can join the network and potentially gain access to
privacy sensitive information on the networking layer, like the IP addresses of peers they are connected to.
The gathered information can then be used to perform deanonymization attacks, e.g. to link transactions to public IP addresses [first described in Koshy]. \\
One of the most important jobs of the network layer in blockchains is to handle broadcast messages, since
both new blocks and new transactions are sent as broadcasts through the network.
While Bitcoin started with a relatively naive approach of broadcasting
new blocks and transactions in the network, by just iteratively spreading the
messages to all nearby nodes,
the network overlay of blockchains has since been revisited multiple times to optimize the message passing with regards to efficiency and privacy [Koshy, Bojja, Fanti, Rohrer, ...] and is continued to be researched and optimized []. \\
One [semi-]recently devoloped peer-to-peer overlay that can be used for efficient broadcasting in P2P networks is Kadcast.
Kadcast is a Kademlia based network overlay that diverges from the naive and redundant approach of ``gossiping'', to a structured message propagation, leveraging the bucket logic introduced by Kademlia, and achieving a complete network coverage with minimal messages (excluding neccessary redundancy for stability reasons). [TODO vllt infografik einfügen wie kadcast/bucket logic funktioniert?]
The initial application for Kadcast was to broadcast newly mined blocks.
Because the direct link between IP addresses and transactions is already
broken when a transaction is included in a block, the block propagation
was optimized for efficiency only, without any concerns for privacy issues. \\
However, when using Kadcast to broadcast transactions instead, an adversary that is connected to the network can potentially perform
deanonymization attacks and link transactions to the public IP addresses that issued them. \\
Since the travel path of messages in Kadcast is structured, and message redundancy is reduced to a minimum, the question was raised, what sorts of privacy implications using Kadcast as a broadcasting algorithm for transactions would have [R.]. \\
To start answering this question, we want to use a quantitative approach and perform a deanonymization attack
as it could happen in a real world scenario. Because of the lack of historical data, we use a [highly] abstracted network
simulation to generate data which then can be used for the attack. The simulation setup, including constraints of our simulation and potential resulting implications for the attack are described in [Section Simulation Setup]. \\
We further model our adversary and describe the means of our attacker, explaining
which simplifications we made to the attacker model, and why we can use such a simplified model
without sacrificing much effectiveness of the attack [Ch. atk.model]. \\
We will then shortly elaborate which metrics we applied to evaluate the impact of the attack [Ch. metrics]. \\
In our evalution we will take a closer look at the results of our simulated attacks, and
review if the privacy impacts of using Kadcast for transaction broadcasting align with our
research hypothesis, that the structured overlay has [negative] effects on the privacy of the network. \\
In the last step of our evalution, we will try to mitigate the privacy issues we have observed previously
by [prefixing /besseres wort] the broadcasts with an anonymity phase as described in [Dandelion]. \\
We conclude our analysis by briefly summarizing our results and giving an outlook
on where future research on this topic could be headed from this point on. \\
TODO Results + Outline



%\section{Evaluation}


%\input{chapters/model.tex}


\end{document}
\endinput
%%
%% End of file `sample-sigconf.tex'.
