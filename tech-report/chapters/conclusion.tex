\section{Conclusion}
The goal of our research was to evaluate which kinds of privacy implications using the Kadcast algorithm for
transaction broadcasting in blockchain based technologies would have. \\
To answer this question we simulated a network which uses Kadcast as a transaction broadcasting algorithm.
To evaluate the privacy properties, we further simulated an attack by a botnet-like adversary that tries
to link transactions to public IP addresses.\\
We launch a simple, but effective deanonymization attack using the so called ``first-spy estimator'', which
classifies the first node that forwards a transaction to the botnet as the originator of said transaction. \\
We have shown that even such a simplistic attack performs relatively well [numbers] [compared to?] in our simulation [how about real world? results applicable?].
Our next step was to try to mitigate the observed privacy issues by
combining Kadcast with a simplified version of the Dandelion Spreading
algorithm introduced by [Fanti]. \\
The experimental results show the clear trend that combining the Kadcast broadcasting phase with a Dandelion anonymity phase
yields privacy benefits. \\
Since our simulation setup is limited and does not account for any node or message failures,
the results of our evalution have to be assessed with caution.
While we find it highly unlikely, that real world experiments would completely contradict our results, we made some strong
assumptions in the attacker model based on what kind of attacks on Kadcast are currently known and our estimated impact of those attacks.
We don't know yet, if there is a realistic attack that e.g. leverages the knowledge an adversary has about the overlay structure,
so that they can perform better deanonymization attacks. \\
An interesting direction for further research on this matter would be to try to enhance the ``first-spy estimator'',
taking into account the characteristic spreading properties of the Kadcast protocol,
and maybe even formally prove to what extend an attacker could exploit this.
Besides reinforcing our observations with a formal model, it would also be benificial to run a more sophisticated network simulation
that incorporates network and node failures (honest or byzantine) and simulates the whole network stack.
