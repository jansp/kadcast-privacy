\section{Related Work\label{related_work}}
The privacy of blockchains is an intensively studied research field.
Deanonymization attacks on different layers of the blockchain have been evaluated [],
e.g. attacks on the consensus layer that try do deanonymize users by linking
different transactions to a single source, even when using different public
identifiers for each transaction []. \\
More recent research adapted general knowledge about P2P networks to the networking layer
of blockchain based technologies and evaluates the privacy properties of the
network overlays of public blockchains [fanti et. al, ...]. \\
There are two papers on which most of our research is directly based on. \\
The first is the work on Kadcast by Rohrer \& Tschorsch[], who proposed the Kadcast
algorithm for broadcasting in P2P networks. While initially proposed for block propagation,
we are more interested in using Kadcast for transaction broadcasting and the privacy implications of such an application.
In that regard our work is a direct follow up to the questions that were
already raised by Rohrer \& Tschorsch[],
and an attempt to investigate some of the potential privacy problems
stated in the original paper. \\
While Kademlia itself is already leveraged by various blockchain based technologies, including Ethereum [],
it isn't commonly used as a broadcasting algorithm. Therefore the privacy properties of such an application
are not well understood yet and there exists neither a formal model nor an empirical evaluation of
the anonymity of Kadcast. \\
The second paper which heavily influenced our work is Dandelion(++)[Fanti et. al], from which we
drew several ideas, especially on how to assess the privacy of transaction broadcasting on the
networking layer. This includes the general idea of the botnet-like
adversary, using precision and recall as anonymity metrics and the first
spy estimator [TODO vllt noch darauf eingehen, wo die sachen wirklich
das erste mal vorgeschlagen wurden]
