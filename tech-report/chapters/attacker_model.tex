\section{Attacker Model\label{attacker_model}}
The overall goal of the attacker in our setup is to deanonymize users in
the network.
More precisely, the attacker wants to map transactions to the public IP addresses that issued them. \\
The Kadcast protocol comes with inherit security properties that mitigate different kind of attacks
and allow us to concisely define the means of an attacker if we assume
those properties hold []. [TODO: short summary of threat model of
kadcast, see kadcast paper] \\
The factor that is most important for mitigating attacks on the network layer, is that participants are
not allowed to choose their position in the overlay freely, and it is instead bound to the public IP address of a peer [].
The IP to ID mapping is done via a cryptographic hash function we assume to be unbreakable.
This makes it considerably harder for attackers to e.g. perform eclipse attacks [] or benefit in other ways from
a freely chosen position in the network. [sybil.] \\
We currently only know of two ways a malicious party that disobeys the
protocol could attack or observe the network differently than an ``honest-but-curious" attacker. \\
Those are namely a denial-of-service attack and an attack in which the adversary tries to flood
benign nodes with ping messages in hope of hitting the time frame in which the benign node has at least one unresponsive
peer, but has yet to refresh its peer list [eclipsing false friends]. In
this case the attacker could potentially insert their own nodes into the
peer list of the benign node.
This attack is made harder in Kadcast because not only does a node need
to have any arbitrary unresponsive peer, but an unresponsive peer
in the same bucket which the adversial node would be placed in. \\
Because of the assumed limited advantages byzantine nodes give an attacker and because our simulation setup doesn't account for any network change and turn,
which would be needed to perform a pinging attack as described above,
we exclude both of these attack vectors from our evalution and instead focus solely on an ``honest-but-curious" attacker. \\
For our analysis we further assume the adversary does not take the
network topology or their overlay-protocol knowledge into account
and only uses information recorded by the spies, including aggregated
and infered information. \\
This also means we assume the adversary does not utilize the broadcasting height attached to each broadcast message
for the deanonymization attack. While we do not formally show the impact
of not using that information,
this assumption can at least be somewhat justified by assuming the broadcasting height can
be hidden or masked to a certain extend (e.g. by sending the same message at different/randomized heights). \\
Under the specified circumstances we found the most realistic scenario
to be a botnet-like adversary that is part of the normal network overlay
and is used to deanonymize other users in the network. \\
We therefore model the attacker as an entity that controls a certain share of nodes in the network (so called ``spies'').
These spies all behave indistinguishably to benign nodes from an outsider's perspective (i.e. they obey the protocol). \\
To perform a deanynomization attack, the adversary logs all messages their nodes receive,
including timestamp and sender (IP address) for each recorded message. After the collection phase,
the attacker feeds the aggregated information up to this point in time into an estimator that
computes macro averaged precision and recall values over the whole dataset as defined in
Section~\ref{metrics}. \\
As an estimator we use the simple ``first-spy-estimator'' [ref].
This estimator maps every transaction to the first IP address that
sent the transaction to any of the spies. \\
Using this setup, it follows that anonymity is evaluated as a
network-wide property. Our evaluation does not incorporate the
perspective of a singular participating user of the network, i.e. we
make no claims on the anonymity of an individual participant of
the network, but only on the anonymity of the network as a whole.
