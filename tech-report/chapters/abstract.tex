When discussing privacy properties of blockchains or Distributed Ledger Technologies, the
focus of the analysis is often on the consensus layer of the blockchain.
Another major factor for a comprehensive privacy analysis is
the network layer, which defines how messages are passed between peers.
On this layer many privacy sensitive informations like public IP addresses of users and connections between peers
are used and can potentially be recorded by malicious parties.
In this report we will conduct a privacy analysis of the structured P2P overlay Kadcast [Rohrer\&Tschorsch],
which can be used for efficient broadcasting in blockchain based technologies.
The focus of our analysis is the usage of Kadcast for transaction broadcasting and the evaluation of an adversary's capability
of linking observed transactions to IP addresses.
We show that Kadcast is susceptible to network wide deanonymization attacks by botnets, corroborating the hypothesis that its
efficient overlay comes at the cost of privacy.
To mitigate the impact of such an attack, we propose to [prefix /besseres wort] the broadcasting phase with an anonymity phase, using
the Dandelion spreading algorithm [Fanti].