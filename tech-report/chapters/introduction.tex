\section{Introduction}
When talking about privacy issues of blockchains or Distributed Ledger Technologies, the
focus of the analysis is often on the consensus and application layers of the blockchain.
Another major factor for a comprehensive analysis of the privacy of blockchains is also
the network layer or network overlay which defines how messages are passed between participants of the network.
On this layer many privacy sensitive informations like the IP addresses of users and the connections between the peers
of the network is used and can potentially be leaked or recorded by malicious parties.
In this report we will do a privacy analysis of the newly developed, Kademlia based,
P2P overlay Kadcast that can be used for efficient message passing in Bitcoin or other blockchain based technologies.

hier evtl. kurz kadcast beschreiben, welche idee das auf privacy zu untersuchen und warum überhaupt, etc.

\section{Simulation Setup}
- discrete event simulation, simpy

\section{Attacker Model}
The Kadcast protocol comes with inherit properties that mitigate different attacks
and allow us to concisely define the means of an attacker if we assume those properties hold.

The main factor that mitigates attacks on the network overlay is that participants are
not allowed to choose their position in the network freely, it is instead bound to the IP address of the peer.
This property makes it considerably harder for attackers to perform eclipse attacks or benefit in other ways from
a freely chosen position in the network overlay.

We currently only know of two ways a byzantine actor could attack or observe the network differently than
an ``honest but curious" attacker.

Those are namely a denial of service attack or so called ``aggressive pinging" [bib], a technique where malicious nodes flood
benign nodes with ping messages in hope of hitting the time window in which the benign node has at least one unresponsive
peer in its peer list(/bucket list), but has yet to refresh its peer list(buckets). In this case
the benign node would replace the unresponsive node in its peer list with the new malicious node.
%react to the ping by exchanging the unresponsive node for the new malicious node in its peer list.


We excluded both of these attack vectors from our evalution (TODO why?) and instead focused solely on an ``honest but curious" attacker (TODO why?).

In our scenario the attacker controls a certain share or fraction of nodes in the network (so called ``spies'').
These spies all behave indistinguishable to benign nodes from an outsider's perspective(/obey the protocol).
The attacker logs all messages their nodes receive(/relay), including timestamps and senders of the messages.

From this point on we have to differentiate between two different kind of attackers: \\
1. An attacker that has no knowledge of the network topology \\
2. An attacker that has knowledge of the network topology

The first part of our evalution focuses solely on scenario 1. and therefore uses only local
network(besseres Wort?) information and information infered by combining the observations of all spie nodes.


In our analysis we fed these informations to an estimator [s.fanti]

- only known options of malicious nodes: (d)dos, aggressive pinging

We therefore chose to 

- different ideas: one supernode connects to all -> not possible because kadkast, botnet -> thats what we use

\section{Evaluation}