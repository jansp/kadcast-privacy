\section{Introduction}
When talking about privacy issues of blockchains or Distributed Ledger Technologies, the
focus of the analysis is often on the consensus and application layers of the blockchain.
Another major factor for a comprehensive analysis of the privacy and anonymity properties
of blockchains is the network layer or network overlay which defines how messages are passed
between participants of the network. On this layer many privacy sensitive informations
like the IP addresses of users and the connections between the peers of the network
is used and can potentially be leaked or recorded by malicious parties.
In this report we will do a privacy analysis of the newly developed, Kademlia based,
P2P overlay Kadcast that can be used for efficient message passing
e.g. in Bitcoin or other blockchain based technologies.

Kadcast was developed as an efficient spreading protocol to propagate blocks in blockchain based infrastructures.
The key difference to e.g. bitcoin like message spreading (also called ``gossiping")
is that a structured P2P overlay is used that allows a coverage of the whole network with minimal messages,
i.e. a protocol is used that makes sure every peer receive a distinct message only once
(or $<n$ times, to account for dropped messages). %TODO besser erklären

hier evtl. kurz kadcast beschreiben, welche idee das auf privacy zu untersuchen und warum überhaupt, etc.

\section{Simulation Setup}
For our simulation we used the discrete event simulation framework ``simpy''
- discrete event simulation, simpy
- no dropped messages/churn -> beta set to 1


\section{Attacker Model}
- common adv. models are: botnet, supernode.
- adv. can: create outbound connections(?), disobey protocol, record traffic (incl. src ip, timestamp, ...)
- attack is network wide, not focused on single peer/target
- goal: create TX-IP mapping
- another assumption: broadcast height can be ``hidden'' (to certain extend)/this information is not available for attacker

The overall goal of the attacker is to deanonymize users in the network.
More precisely, the attacker wants to map certain transactions to certain IP addresses.

The Kadcast protocol comes with inherit security properties that mitigate different kind of attacks
and allow us to concisely(/better) define the means of an attacker if we assume those properties hold [].

The main factor that mitigates attacks on the network layer is that participants are
not allowed to choose their position in the network overlay freely, because it is bound to the IP address of a peer [].
This property makes it considerably harder for attackers to perform eclipse attacks [] or benefit in other ways from
a freely chosen position (TODO which other ways?).

We currently only know of two ways an actor that disobeys the protocol could attack or observe the network differently than
an ``honest but curious" attacker.

Those are namely a denial of service attack or so called ``aggressive pinging" [bib], a technique where malicious nodes flood
benign nodes with ping messages in hope of hitting the time window in which the benign node has at least one unresponsive
peer in its peer list(/bucket list), but has yet to refresh its peer list(buckets). In this case
the benign node would replace the unresponsive node in its peer list with the new malicious node.
%react to the ping by exchanging the unresponsive node for the new malicious node in its peer list.

Because of the limited possibilities an attacker has and because our simulation setup (TODO explain further...)
we excluded both of these attack vectors from our evalution and instead focused solely on an ``honest but curious" attacker.

In our setup the attacker controls a certain share(/fraction) of nodes of the network (so called ``spies'').
These spies all behave indistinguishable to benign nodes from an outsider's perspective(/obey the protocol).
The attacker logs all messages their nodes receive(/relay), including timestamps and senders(/IP adr) of the messages.

From this point on we have to differentiate between two different kind of attackers: \\
1. An attacker that has no knowledge of the network topology \\
2. An attacker that has knowledge of the network topology

The first part of our evalution focuses solely on scenario 1. and therefore uses only local
network(besseres Wort?) information and information infered by combining the observations of all spie nodes.


In our analysis we fed these informations to an estimator [s.fanti]

- only known options of malicious(byzantine) nodes: (d)dos, aggressive pinging

We therefore chose to 

- different ideas: one supernode connects to all -> not possible because kadkast, botnet -> thats what we use

\section{Evaluation}